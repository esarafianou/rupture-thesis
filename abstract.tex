\begin{abstractgr}

Η ασφάλεια είναι ένα από τα βασικά χαρακτηριστικά κάθε υπολογιστικού
συστήματος, καθώς εξασφαλίζει την εμπιστευτικότητα, την ακεραιότητα και τη διαθεσιμότητα
της πληροφορίας. Η παρούσα εργασία ερευνά επιθέσεις πάνω σε συμπιεσμένα
κρυπτογραφημένα πρωτόκολλα και συγκεκριμένα αναπτύσσει περαιτέρω την επίθεση BREACH.

Προτίνονται στατιστικές μέθοδοι που επιτρέπουν να πραγματοποιηθεί
η επίθεση σε ιστοσελίδες που χρησιμοποιουν συγχρονες κρυπτογραφικες
τεχνικες. Επιπλεον, αναπτυχθηκαν τεχvικές βελτιστοποίησης της επίθεσης που συντομεύουν
την επιθέση εως και 500 φορές σε θεωρητικό επίπεδο.

Δημηιουργήσαμε ένα εργαλείο, το Rupture, που εφαρμόζει τις στατιστικές μας μεθόδους και
τις τεχνικές βελτιστοποίησης. Το Rupture ειναι γραμμενο σε κωδικα επιπέδου παραγωγής 
ελευθερου λογισμικού και αυτοματοποιει την επιθεση. 
Το RESTful API που εκθέτει σε συνδυασμό με το Web User Interface το καθιστούν
ένα εύχρηστο εργαλείο που επιτρέπει στους επιτιθέμενους να πραγματοποιούν μαζικές 
επιθέσεις.

Η δημιουργία αυτού του εργαλείου δε στοχεύει σε καμία περίπτωση σε κακόβουλη χρήση.
Αντίθετα, επιδιώκει να ευαισθητοποιήσει την κοινότητα για τις επιθέσεις σε συμπιεσμένα
κρυπτογραφημένα πρωτόκολλα. Αυτες οι επιθέσεις ειναι αρκετα εκλεπτυσμένες, με αποτέλεσμα
να θεωρείται αμφίβολο το αν μπορούν να πραγματοποιηθούν σε συνθήκες πραγματικού κόσμου. 

Η δημίουργια του Rupture ειναι συνεργατικη με τους (αλφαβητική σειρά): Δημήτρη Γρηγορίου, 
Δημήτρη Καρακώστα, Διονύση Ζήνδρο.
Ανανεωμένες εκδόσεις της παρούσας εργασίας μπορούν να βρεθούν στον ακόλουθο
σύνδεσμο: \url{https://github.com/esarafianou/rupture-thesis}. Το εργαλείο
Rupture βρίσκεται στο σύνδεσμο: \url{https://github.com/dionyziz/rupture}.

\end{abstractgr}

\begin{abstracten}

Security is a fundamental aspect of every computer system, as it reassures
 confidentiality, integrity, and availability of the data. This work investigates
attacks on compressed encrypted protocols, and develops further the BREACH attack.

New statistical methods are being proposed, which allow the attack to perform in real life websites
which use the latest cryptographic techniques. New optimizations of the attack were also developed, which speed
up the attack up to 500 times in the theory.

We implemented a framework, Rupture, which applies our statistical methods and 
optimization techniques. Rupture is open-source, production level and automates the attack.
The RESTful API it exposes in combination with the Web User Interface make it an easy-to-use tool 
which allows the attackers to perform attacks in the wild.

The implementation of this framework focuses by no means in malicious purposes. On the contrary,
it aims to sensitize the community about compression side-channel attacks. Such attacks are quite 
sophisticated and the community doubts whether the could be real life attacks

Rupture is a collaborative work with (alphabetical order): Dimitris Grigoriou, Dimitris Karakostas
and Dionysis Zindros.

Updated versions of the current work can be found on the following link:
\url{https://github.com/esarafianou/rupture-thesis}. Rupture repository is:
\url{https://github.com/dionyziz/rupture}

\end{abstracten}

\begin{acknowledgementsgr}
Η παρούσα διπλωματική εργασία εκπονήθηκε στα πλαίσια της φοίτησής μου στο τμήμα
Ηλεκτρολόγων Μηχανικών και Μηχανικών Υπολογιστών του Εθνικού Μετσόβιου
Πολυτεχνείου.

Η διπλωματική αυτή εκπονήθηκε υπό την επίβλεψη του καθηγητή Αριστείδη Παγουρτζή,
τον οποίο θα ήθελα να ευχαριστήσω θερμά για τη βοήθειά του, καθώς και για το
γεγονός ότι μέσω της διδασκαλίας της Κρυπτογραφίας με εισήγαγε
και μου ενεπνευσε την αγαπη για το αντικείμενο.

Ακόμα, θα ήθελα να ευχαριστήσω τον Διονύση Ζήνδρο, για την αμέριστη βοηθεια και καθοδηγηση του
στο επίπεδο της εκπόνησης της εργασιας, τον Δημήτρη Καρακώστα για τις συμβουλές και
τις διορθώσεις του στον κώδικα καθώς και τον Πέτρο Αγγελάτο για την πρακτική και συναισθηματική υποστήριξή του.

Τέλος, θα ήθελα να ευχαριστήσω τους φίλους και την οικογένειά μου για τη στήριξη
που μου παρείχαν όλα αυτά τα χρόνια.

\end{acknowledgementsgr}
