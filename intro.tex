\chapter{Εισαγωγή}\label{ch:intro}

\section{Εισαγωγή}\label{sec:intro}

Τα τελευταία χρόνια παρατηρείται μια αύξηση του ενδιαφέροντος της κοινωνίας
σχετι- κά με τα θέματα της ασφάλειας και της ιδιωτικότητας της πληροφορίας στο
Διαδίκτυο. Οι αποκαλύψεις του Edward Snowden, η διαρροή προσωπικών στοιχείων και
κωδικών χρηστών μεγάλων εταιρειών, όπως yahoo και dropbox άλλαξαν τον τρόπο
με τον οποίο αντιλαμβανόμαστε τη χρήση online υπηρεσιών. Οι ερευνητές και
χρήστες στράφηκαν στην αναζήτηση λύσεων ώστε οι επικοινωνίες να γίνουν πιο
ασφαλείς απέναντι σε κάθε είδους αντιπάλους.

Στην παρούσα εργασία στοχεύουμε να αναδείξουμε αδυναμίες στα πρωτόκολλα
που χρησιμοποιούνται στο διαδίκτυο και μέσω της δημοσιευσής της να 
ευαισθητοποιήσουμε την κοινότητα ώστε να αντιμετωπιστούν αποτελεσματικά αυτές
οι ευπάθειες.

Η έρευνά μας επικεντρώνεται σε επιθέσεις ενάντια σε πρωτόκολλα συμπίεσης που
εφαρμόζονται στα δεδομένα πριν την κρυπτογράφηση και μεταφορά τους. Συγκεκριμέ- να,
επεκτείνουμε υπάρχοντα μοντέλα επίθεσης, όπως το BREACH, και δημιουργούμε ένα εργαλείο
που πραγματοποιεί με αυτοματοποιημένο τρόπο τέτοιες επιθέσεις.

Το λογισμικό συμπίεσης στο οποίο επικεντρωθήκαμε είναι το gzip, που 
χρησιμοποιείται ευρέως στο Διαδίκτυο. Το gzip εφαρμόζει τον αλγόριθμο
DEFLATE, που αποτελεί τον συνδυασμό των αλγορίθμων συμπίεσης Huffman
και LZ77. Η επίθεση που επεκτείνουμε εκμεταλλεύεται τον τρόπο με τον
οποίο συμπιέζει κείμενα το LZ77 ενώ αντίθετα ο αλγόριθμος Huffman
δυσκολεύει την επίθεση.

Το πιο διαδεδομένο πρωτόκολλο μεταφοράς δεδομένων στο Διαδίκτυο είναι το
HTTP (Hyper-Text-Transfer Protocol). Τα δεδομένα που μεταφέρονται μέσω HTTTP
δεν είναι κρυπτογραφημένα, με αποτέλεσμα οποιοσδήποτε επιτιθέμενος ελέγχει
το δίκτυο μας, να μπορει να διαβάσει και να αλλοιώσει τα δεδομένα που μεταφέρονται.
Η εξέλιση του HTTP, το HTTPS, καλύπτει αυτό το κενό στην ασφάλεια με την εισαγωγή
ενός ακόμα επιπέδου δικτύου πριν το επίπεδο εφαρμογής που αρχικά ήταν το SSL 
(Secure Socket Layer) και στη συνέχεια το TLS (Transport Layer Security). Το επίπεδο αυτό
επιβάλλει την κρυπτογράφηση των δεδομένων πριν  απο τη μεταφορά τους και εγγυάται
την εμπιστευτικότητα, ακεραιότητα και διαθεσιμότητα των δεδομένων.

Οι αλγόριθμοι κρυπτογράφησης που χρησιμοποιούνται σε αυτό το επίπεδο μπορούν να
χωριστούν σε δύο μεγάλες κατηγορίες: αλγόριθμοι ροής και αλγόριθμοι δέσμης.
Στην πρώτη περίπτωση, τα δεδομένα κρυπτογραφούνται ως μια συνεχής ροή,  ενώ στη 
δεύτερη περίπτωση χωρίζονται σε δέσμες ίσου μεγέθους και κρυπτογραφείται κάθε δέσμη χωριστά. 
Σε περίπτωση που τα δεδομένα δεν κατανέμονται με ακρίβεια σε δέσμες, εισάγεται τεχνητός
θόρυβος ώστε να επιτευχθεί το επιθυμητό μέγεθος και να ευθυγραμμιστεί η κάθε δέσμη.

Ο κυριότερος αλγόριθμος ροής ειναι ο RC4\cite{rc4}, ο οποίος όμως δεν χρησιμοποιείται πλέον
γιατι εχουν βρεθει σημαντικές ευπάθειες που τον καθιστούν ανασφαλή. Ο πιο
διαδεδομένος αλγόριθμος δέσμης ειναι ο AES, που σε διάφορες παραλλαγές είναι ο
πιο ευρέως χρησιμοποιούμενος αλγόριθμος κρυπτογράφησης. Η χρήση αλγορίθμων δέσμης
δυσκολευουν την σημαντικα την επιθεση που περιγράφουμε συγκριτικά με τους αλγόριθμους ροής.
Ωστόσο, με την παρούσα εργασία καταδεικνύουμε οτι οι αλγό- ριθμοι δέσμης όπως ο AES δε
μας προστατεύουν απόλυτα από τέτοιες επιθέσεις. 

Προκειμένου να πετύχουμε την επίθεση στον αλγόριθμο AES χρησιμοποιήσαμε στατι- στικές
μεθόδους και τεχνητό θόρυβο για να ξεπεράσουμε τις δυσκολίες που εισαγονται από τη
χρήση δεσμών. Ταυτόχρονα εισάγαμε τεχνικές που βελτιστοποιούν την απόδοση της
επίθεσης και μας επιτρέπουν να την πραγματοποιούμε σε ρεαλιστικούς χρόνους.

Αναπτύξαμε ανοιχτό λογισμικό σε κώδικα επιπέδου παραγωγής που εφαρμόζει τις στατιστικές
μεθόδους και τεχνικές βελτιστοποίησης μας. Το λογισμικό είναι ενα σύστη- μα server-based
αρχιτεκτονικής που αυτοματοποιεί την επίθεση καθώς απαιτεί μικρή διαμόρφωση πριν από την
επίθεση. Δίνει τη δυνατότητα για πολλαπλές επιθέσεις την ίδια στιγμή σε διαφορετικά θύματα/χρήστες
και ιστοσελίδες. Με το λογισμικό αυτό, πραγματο- ποιήσαμε επιθεσεις σε εργαστηριακές ιστοσελίδες.

Όσον αφορά τις στατιστικές μεθόδους, η χρήση πιθανοτητων στην επίθεση μας την καθιστά μη
ντετερμινιστική. Προκειμένου λοιπόν να έχουμε ακρίβεια στα αποτελέσματα μας, έχουμε εισάγει
μια μετρική που εκφραζει την αυτοπεποιθησή μας για τα αποτελέσματα που παραγονται απο
τις στατιστικές μας μεθόδους. Μόνο αν αυτή ειναι ικανή, θεωρούνται έγκυρα τα δεδόμένα μας.
Σε διαφορετική περίπτωση, απορρίπτονται.


Εν κατακλείδι, η παρούσα εργασία αποτελεί τη συνέχεια μια ομάδας ερευνών που
παρουσιάστηκαν τα τελευταία χρόνια και φανέρωσαν βασικές αδυναμίες στα
συστήματα που χρησιμοποιούμε κατά κόρον. Είναι σημαντικό να επεκταθεί με νέες
τεχνικές βελτιστο- ποίησης της επίθεσης και, κυρίως, νέες μεθόδους αντιμετώπισής
της.

\section{Δομή της εργασίας}\label{sec:structure}

Η εργασία έχει δομηθεί ως εξής:

 
\begin{description} \item{Κεφάλαιο \ref{background}} \hfill \\

Το κεφάλαιο αυτό παρέχει στον αναγνώστη βασικές πληροφορίες, τόσο σε τεχνικό όσο
και σε θεωρητικό επίπεδο, οι οποίες θα χρησιμοποιηθούν στη συνέχεια.
Θα περιγράψουμε τους πιο διαδεδομένους αλγόριθμους συμπίεσης, καθώς και βασικά
πρωτόκολλα που χρησιμοποιούνται για την ασφάλεια στις επικοινωνίες, καθώς και
επιθέσεις εναντίων τους.\hfill \\

\item{Κεφάλαιο \ref{statistic_methods}} \hfill \\
 
Το κεφάλαιο αυτό εισαγει στατιστικές μεθόδους για να παρακαμφθουν εμπόδια
και να γίνει δυνατή η επίθεση σε αλγόριθμους δέσμης. Προτείνονται ακόμα μέθoδοι
βελτιστοποίησης που αυξάνουν την ταχύτητα και την απόδοση της επίθεσης.


\item{Κεφάλαιο \ref{rupture}} \hfill \\

Σ' αυτο το κεφάλαιο περιγράφουμε το εργαλείο που δημιουργήσαμε για επιθέσεις
σε συμπιεσμένα κρυπτογραφημένα πρωτόκολλα. Συγκεκριμένα, επισημαίνουμε τις
προυποθέσεις κατω απ τις οποίες πραγματοποιείται μια τέτοια επίθεση, αναλύ- ουμε
την ορολογία που χρησιμοποιείται στο εργαλειο και περιγραφουμε διεξοδικα τα
διάφορα μέρη του και τις λειτουργίες που αυτά επιτελούν.


\item{Κεφάλαιο \ref{rupture_api}} \hfill \\

Το κεφάλαιο αυτό περιγράφει πώς επεκτείναμε το εργαλείο μας ώστε να γίνει πιο
λειτουργικό και εύχρηστο. Περιγράφει το RESTful API που δημιουργήσαμε και τις
κλήσεις που γίνονται σ'αυτο καθώς και τη λειτουργικότητα της διεπαφής χρήστη
(User Interface).


\item{Κεφάλαιο \ref{appendix}} \hfill \\

Ο κώδικας υλοποίησης της επίθεσης.  

\end{description}

